% Created 2021-04-19 Mon 12:49
% Intended LaTeX compiler: pdflatex
\documentclass[11pt]{article}
\usepackage[utf8]{inputenc}
\usepackage[T1]{fontenc}
\usepackage{graphicx}
\usepackage{grffile}
\usepackage{longtable}
\usepackage{wrapfig}
\usepackage{rotating}
\usepackage[normalem]{ulem}
\usepackage{amsmath}
\usepackage{textcomp}
\usepackage{amssymb}
\usepackage{capt-of}
\usepackage{hyperref}
\usepackage[top=0in, bottom=1.25in, left=1.25in, right=1.25in]{geometry}

\usepackage[main,include]{embedall}
\IfFileExists{./\jobname.org}{\embedfile[desc=The original file]{\jobname.org}}{}
\author{Jake Moss}
\date{\today}
\title{Computer Graphics Proposal}
\hypersetup{
 pdfauthor={Jake Moss},
 pdftitle={Computer Graphics Proposal},
 pdfkeywords={},
 pdfsubject={},
 pdfcreator={Emacs 28.0.50 (Org mode 9.5)}, 
 pdflang={English}}
\begin{document}

\maketitle

\section{What?}
\label{sec:orgd154424}
In this project I aim to animate a functional 3D chess board using various computer graphics techniques.

I will be using python and opengl to produce and animate a functional chess board capable of playing a game or stepping through any game provided. To provide the chess functionality I will use the \texttt{python-chess} library to handle the complex move validation.

I will be writing the slide and capture animations, and constructing the models using blender.

I plan to add the following features
\begin{itemize}
\item Board
\item Piece models
\item Piece animations
\item UI to play a game from a SAN string, reset, and see move history
\item Click and drag pieces to new positions
\item Highlight possible move lines.
\item Use an engine to evaluate board states - Maybe
\item CPU player - Maybe
\end{itemize}
\section{Why?}
\label{sec:org7c371ae}
The purpose of this project is to gain a further understanding of the structure and process of creating a computer graphics program as well as efficient techniques and methods while in the process of creating a chess program.
\section{How?}
\label{sec:org1776a0e}
I will use blender to create the piece models so that they can be visually appealing and unique.
\begin{itemize}
\item \texttt{PyOpenGl} for animations and rendering.
\item \texttt{python-chess} for the chess specifics such as move validation and interfacing with engines.
\end{itemize}
\end{document}
